% LaTeX Template für Abgaben an der Universität Stuttgart
% Autor: Sandro Speth
% Bei Fragen: Sandro.Speth@iste.uni-stuttgart.de
%-----------------------------------------------------------
% Hauptmodul des Templates: Hier können andere Dateien eingebunden werden
% oder Inhalte direkt rein geschrieben werden.
% Kompiliere dieses Modul um eine PDF zu erzeugen.

% Dokumentenart. Ersetze 12pt, falls die Schriftgröße anzupassen ist.
\documentclass[12pt]{scrartcl}
% Einbinden der Pakete, des Headers und der Formatierung.
% Mit den \include und \input Befehlen können Dateien eingebunden werden:
% \include: Fügt einen Seitenumbruch nach dem Text ein
% \input: Fügt KEINEN Seitenumbruch nach dem Text ein
\usepackage[utf8]{inputenc}
\usepackage[T1]{fontenc}
\usepackage[ngerman]{babel}
\usepackage{csquotes}
\usepackage{lmodern}
\usepackage{graphicx}
\usepackage{float}
\usepackage[pdftex,hyperref,dvipsnames]{xcolor}
\usepackage{listings}
\usepackage[a4paper,lmargin={2cm},rmargin={2cm},tmargin={3.5cm},bmargin = {2.5cm},headheight = {4cm}]{geometry}
\usepackage{amsmath,amssymb,amstext,amsthm}
\usepackage[lined,algonl,boxed]{algorithm2e}
\usepackage{tikz}
\usepackage{hyperref}
\usepackage{url}
\usepackage[inline]{enumitem} % Ermöglicht ändern der enum Item Zahlen
\usepackage[headsepline]{scrlayer-scrpage}
\usepackage{booktabs}
\usepackage{multirow}
\usepackage{pdflscape}
\usepackage{diagbox}
\pagestyle{scrheadings}
\usetikzlibrary{automata,positioning}
% LaTeX Template für Abgaben an der Universität Stuttgart
% Autor: Sandro Speth
% Bei Fragen: Sandro.Speth@iste.uni-stuttgart.de
%-----------------------------------------------------------
% Modul beinhaltet Befehl fuer Aufgabennummerierung,
% sowie die Header Informationen.

\providecommand{\BlattNr}{0}
\providecommand{\AuthorsNames}{keine Autoren gesetzt}

% Überschreibt enumerate Befehl, sodass 1. Ebene Items mit
\renewcommand{\theenumi}{(\alph{enumi})}
% (a), (b), etc. nummeriert werden.
\renewcommand{\labelenumi}{\text{\theenumi}}

% Counter für das Blatt und die Aufgabennummer.
% Ersetze die Nummer des Übungsblattes und die Nummer der Aufgabe
% den Anforderungen entsprechend.
% Gesetz werden die counter in der hauptdatei, damit siese hier nicht jedes mal verändert werden muss
% Beachte:
% \setcounter{countername}{number}: Legt den Wert des Counters fest
% \stepcounter{countername}: Erhöht den Wert des Counters um 1.
\newcounter{exnum}

% Befehl für die Aufgabentitel
\newcommand{\exercise}[1]{\section*{Aufgabe \theexnum\stepcounter{exnum}: #1}} % Befehl für Aufgabentitel

% Formatierung der Kopfzeile
% \ohead: Setzt rechten Teil der Kopfzeile mit
% Namen und Matrikelnummern aller Bearbeiter
\ohead {\AuthorsNames}
% \chead{} kann mittleren Kopfzeilen Teil sezten
% \ihead: Setzt linken Teil der Kopfzeile mit
% Modulnamen, Semester und Übungsblattnummer
\ihead{Advanced Software Engineering\\
Wintersemester 2025\\
Übungsblatt \BlattNr}
\renewcommand{\arraystretch}{1.5} 
\usepackage{pdflscape}
\setcounter{exnum}{1} % Nummer der Aufgabe

\begin{document}
\providecommand{\BlattNr}{0}
\providecommand{\AuthorsNames}{keine Authoren gesetzt}

\begin{titlepage}
	\centering
	\includegraphics[width=0.35\textwidth]{../assets/icon_ase.png}\par\vspace{1cm}
	{\scshape\LARGE Universität Stuttgart \par}
	\vspace{1cm}
	{\scshape\Large Advanced Software Engineering\par}
	\vspace{1.5cm}
	{\huge\bfseries Übungsblatt \BlattNr \par}
	\vspace{2cm}
	{\Large \AuthorsNames \par}
	\vfill
    Wintersemester 2025
    \vfill
	{\large \today\par}
	\end{titlepage}
 % Titelseite einbinden
\section*{\underline{Aufgabe 1:}}
\begin{enumerate}[label=\alph*)]
  \item ~ \\ \begin{tabular}{|c|c|c|}
    \hline
    \textbf{Äquivalenzklasse} & ~~~\textbf{Status}~~~ & \textbf{Beschreibung}                  \\
    \noalign{\hrule height 1.2pt}
    \noalign{\vskip 1pt}
    \noalign{\hrule height 1.2pt}
    $ÄK_1$~~~                    & ~~~gültig~~~          & Status == GEPLANT                      \\
    \hline
    $ÄK_2$~~~                    & ~~~ungültig~~~        & Status != GEPLANT (status == ABGESAGT) \\
          \hline
          $ÄK_3$~~~                    & ~~~gültig~~~          & 0 $\leq$ freie Plaetze $\leq$ 500       \\
          \hline
          $ÄK_4$~~~                    & ~~~ungültig~~~        & freie Plaetze $<$ 0                     \\
          \hline
          $ÄK_5$~~~                    & ~~~ungültig~~~        & freie Plaetze $>$ 500                   \\
          \hline
          $ÄK_6$~~~                    & ~~~gültig~~~          & 1 $\leq$ Anzahl Tickets $\leq$ 6       \\
          \hline
          $ÄK_7$~~~                    & ~~~ungültig~~~        & Anzahl Tickets $<$ 1                   \\
          \hline
          $ÄK_8$~~~                    & ~~~ungültig~~~        & Anzahl Tickets $>$ 6                   \\
          \hline
          $ÄK_9$~~~                    & ~~~gültig~~~          & freie Plaetze $\geq$ Anzahl Tickets     \\
          \hline
          $ÄK_{10}$~~~                   & ~~~ungültig~~~        & freie Plaetze $<$ Anzahl Tickets        \\
          \hline
        \end{tabular} 
        ~\\\\\\\\\\\\
        Die Teilaufgabe b) befindet sich auf der nächsten Seite.
        \newpage
        \end{enumerate}

        \begin{landscape}
        b) ~ \\ \begin{tabular}{|c|c|c|c|c|c|c|}
          \hline
          \textbf{Testklasse} & \textbf{Status} & \textbf{freiePlaetze} & \textbf{anzahlTickets} & \textbf{Erwartetes Ergebnis} & \textbf{Beschreibung Testcase} & \textbf{ÄK} \\
          \noalign{\hrule height 1.2pt}
          \noalign{\vskip 1pt}
          \noalign{\hrule height 1.2pt}
          $TK_1$                & GEPLANT         & 1                     & 1                      & true                         & Min. Tickets, Min. Plätze      & 1, 3, 6, 9            \\
          \hline
          $TK_2$                & GEPLANT         & 500                   & 6                      & true                         & Max. Plätze, Max. Tickets      & 1, 3, 6, 9            \\
          \hline
          $TK_3$                & GEPLANT         & 6                     & 6                      & true                         & Max. Tickets, passend          & 1, 3, 6, 9            \\
          \hline
          $TK_4$                & ABGESAGT        & 100                   & 2                      & false                        & Status falsch                  & 2, 3, 6, 9            \\
          \hline
          $TK_5$                & GEPLANT         & -1                    & 1                      & false                        & Plätze edgecase min.           & 1, 4, 6, 10           \\
          \hline
          $TK_6$                & GEPLANT         & -50                   & 5                      & false                        & Plätze unter min.              & 1, 4, 6, 10           \\
          \hline
          $TK_7$                & GEPLANT         & 501                   & 3                      & false                        & Plätze edgecase max.           & 1, 5, 6, 9            \\
          \hline
         $ TK_8 $               & GEPLANT         & 600                   & 4                      & false                        & Plätze über max.               & 1, 5, 6, 9            \\
          \hline
          $TK_9$                & GEPLANT         & 400                   & 0                      & false                        & anzahlTickets edgecase min.    & 1, 3, 7, 9            \\
          \hline
         $ TK_{10}$               & GEPLANT         & 400                   & -25                    & false                        & anzahlTickets unter min.       & 1, 3, 7, 9            \\
          \hline
         $ TK_{11}$               & GEPLANT         & 400                   & 7                      & false                        & anzahlTickets edgecase max.    & 1, 3, 8, 9            \\
          \hline
          $TK_{12}$               & GEPLANT         & 4                     & 5                      & false                        & freiePlaetze < anzahlTickets   & 1, 3, 6, 10           \\
          \hline
        \end{tabular}
\end{landscape}

% Nutze den \exercise{Aufgabenname} Befehl, um eine neue Aufgabe zu beginnen.
% Möchtest du eine Aufgabe in der Nummerierung überspringen, schreibe vor der Aufgabe: \stepcounter{exnum}
% Möchtest du die Nummer einer Aufgabe auf eine beliebige Zahl x setzen, schreibe vor der Aufgabe: \setcounter{exnum}{x}

\end{document}