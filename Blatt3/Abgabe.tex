% LaTeX Template für Abgaben an der Universität Stuttgart
% Autor: Sandro Speth
% Bei Fragen: Sandro.Speth@iste.uni-stuttgart.de
%-----------------------------------------------------------
% Hauptmodul des Templates: Hier können andere Dateien eingebunden werden
% oder Inhalte direkt rein geschrieben werden.
% Kompiliere dieses Modul um eine PDF zu erzeugen.

% Dokumentenart. Ersetze 12pt, falls die Schriftgröße anzupassen ist.
\documentclass[12pt]{scrartcl}
% Einbinden der Pakete, des Headers und der Formatierung.
% Mit den \include und \input Befehlen können Dateien eingebunden werden:
% \include: Fügt einen Seitenumbruch nach dem Text ein
% \input: Fügt KEINEN Seitenumbruch nach dem Text ein
\usepackage[utf8]{inputenc}
\usepackage[T1]{fontenc}
\usepackage[ngerman]{babel}
\usepackage{csquotes}
\usepackage{lmodern}
\usepackage{graphicx}
\usepackage{float}
\usepackage[pdftex,hyperref,dvipsnames]{xcolor}
\usepackage{listings}
\usepackage[a4paper,lmargin={2cm},rmargin={2cm},tmargin={3.5cm},bmargin = {2.5cm},headheight = {4cm}]{geometry}
\usepackage{amsmath,amssymb,amstext,amsthm}
\usepackage[lined,algonl,boxed]{algorithm2e}
\usepackage{tikz}
\usepackage{hyperref}
\usepackage{url}
\usepackage[inline]{enumitem} % Ermöglicht ändern der enum Item Zahlen
\usepackage[headsepline]{scrlayer-scrpage}
\usepackage{booktabs}
\usepackage{multirow}
\usepackage{pdflscape}
\usepackage{diagbox}
\pagestyle{scrheadings}
\usetikzlibrary{automata,positioning}
% LaTeX Template für Abgaben an der Universität Stuttgart
% Autor: Sandro Speth
% Bei Fragen: Sandro.Speth@iste.uni-stuttgart.de
%-----------------------------------------------------------
% Modul beinhaltet Befehl fuer Aufgabennummerierung,
% sowie die Header Informationen.

\providecommand{\BlattNr}{0}
\providecommand{\AuthorsNames}{keine Autoren gesetzt}

% Überschreibt enumerate Befehl, sodass 1. Ebene Items mit
\renewcommand{\theenumi}{(\alph{enumi})}
% (a), (b), etc. nummeriert werden.
\renewcommand{\labelenumi}{\text{\theenumi}}

% Counter für das Blatt und die Aufgabennummer.
% Ersetze die Nummer des Übungsblattes und die Nummer der Aufgabe
% den Anforderungen entsprechend.
% Gesetz werden die counter in der hauptdatei, damit siese hier nicht jedes mal verändert werden muss
% Beachte:
% \setcounter{countername}{number}: Legt den Wert des Counters fest
% \stepcounter{countername}: Erhöht den Wert des Counters um 1.
\newcounter{exnum}

% Befehl für die Aufgabentitel
\newcommand{\exercise}[1]{\section*{Aufgabe \theexnum\stepcounter{exnum}: #1}} % Befehl für Aufgabentitel

% Formatierung der Kopfzeile
% \ohead: Setzt rechten Teil der Kopfzeile mit
% Namen und Matrikelnummern aller Bearbeiter
\ohead {\AuthorsNames}
% \chead{} kann mittleren Kopfzeilen Teil sezten
% \ihead: Setzt linken Teil der Kopfzeile mit
% Modulnamen, Semester und Übungsblattnummer
\ihead{Advanced Software Engineering\\
Wintersemester 2025\\
Übungsblatt \BlattNr}
\hbadness=99999 % No fucking underfull hbox warning
\hfuzz=9999pt % No fucking overfull hbox warning
\usetikzlibrary{arrows.meta, positioning, calc, shapes.geometric}
\renewcommand{\arraystretch}{1.5}
\newcommand{\textarrow}[1]{%
    \tikz[baseline, inner sep=0pt] {
        % 1. Create a hidden node with the text to measure width
        \node[anchor=south, font=\scriptsize, align=center, inner ysep=2pt] (txt) at (0,0) {#1};
        % 2. Draw the arrow underneath it
        \draw[->, >=Latex, thick, rounded corners] 
             ($(txt.south west)$) -- ($(txt.south east) + (3mm, 0)$);
    }%
}

\setcounter{exnum}{1} % Nummer der Aufgabe

\begin{document}
\providecommand{\BlattNr}{0}
\providecommand{\AuthorsNames}{keine Authoren gesetzt}

\begin{titlepage}
	\centering
	\includegraphics[width=0.35\textwidth]{../assets/icon_ase.png}\par\vspace{1cm}
	{\scshape\LARGE Universität Stuttgart \par}
	\vspace{1cm}
	{\scshape\Large Advanced Software Engineering\par}
	\vspace{1.5cm}
	{\huge\bfseries Übungsblatt \BlattNr \par}
	\vspace{2cm}
	{\Large \AuthorsNames \par}
	\vfill
    Wintersemester 2025
    \vfill
	{\large \today\par}
	\end{titlepage}
 % Titelseite einbinden
\exercise{Funktionsorientierter Test – Online-Ticketsystem}
\begin{enumerate}[label=\alph*)]
  \item ~ \\
        \begin{tabular}{|c|c|c|}
          \hline
          \textbf{Äquivalenzklasse} & ~~~\textbf{Status}~~~ & \textbf{Beschreibung}                  \\
          \noalign{\hrule height 1.2pt}
          \noalign{\vskip 1pt}
          \noalign{\hrule height 1.2pt}
          $\text{ÄK}_1$~~~          & ~~~gültig~~~          & Status == GEPLANT                      \\
          \hline
          $\text{ÄK}_2$~~~          & ~~~ungültig~~~        & Status != GEPLANT (status == ABGESAGT) \\
          \hline
          $\text{ÄK}_3$~~~          & ~~~gültig~~~          & 0 $\leq$ freie Plaetze $\leq$ 500      \\
          \hline
          $\text{ÄK}_4$~~~          & ~~~ungültig~~~        & freie Plaetze $<$ 0                    \\
          \hline
          $\text{ÄK}_5$~~~          & ~~~ungültig~~~        & freie Plaetze $>$ 500                  \\
          \hline
          $\text{ÄK}_6$~~~          & ~~~gültig~~~          & 1 $\leq$ Anzahl Tickets $\leq$ 6       \\
          \hline
          $\text{ÄK}_7$~~~          & ~~~ungültig~~~        & Anzahl Tickets $<$ 1                   \\
          \hline
          $\text{ÄK}_8$~~~          & ~~~ungültig~~~        & Anzahl Tickets $>$ 6                   \\
          \hline
          $\text{ÄK}_9$~~~          & ~~~gültig~~~          & freie Plaetze $\geq$ Anzahl Tickets    \\
          \hline
          $\text{ÄK}_{10}$~~~       & ~~~ungültig~~~        & freie Plaetze $<$ Anzahl Tickets       \\
          \hline
        \end{tabular}
        ~\\\\\\\\\\\\
        Die Teilaufgabe b) befindet sich auf der nächsten Seite.
        \newpage
\end{enumerate}

\begin{landscape}
  b) ~ \\\\
  \begin{tabular}{|c|c|c|c|c|c|c|}
    \hline
    \textbf{Testklasse} & \textbf{Status} & \textbf{freiePlaetze} & \textbf{anzahlTickets} & \textbf{Erwartetes Ergebnis} & \textbf{Beschreibung Testcase} & \textbf{ÄK} \\
    \noalign{\hrule height 1.2pt}
    \noalign{\vskip 1pt}
    \noalign{\hrule height 1.2pt}
    $TK_1$              & GEPLANT         & 1                     & 1                      & true                         & Min. Tickets, Min. Plätze      & 1, 3, 6, 9  \\
    \hline
    $TK_2$              & GEPLANT         & 500                   & 6                      & true                         & Max. Plätze, Max. Tickets      & 1, 3, 6, 9  \\
    \hline
    $TK_3$              & GEPLANT         & 6                     & 6                      & true                         & Max. Tickets, passend          & 1, 3, 6, 9  \\
    \hline
    $TK_4$              & ABGESAGT        & 100                   & 2                      & false                        & Status falsch                  & 2, 3, 6, 9  \\
    \hline
    $TK_5$              & GEPLANT         & -1                    & 1                      & false                        & Plätze edgecase min.           & 1, 4, 6, 10 \\
    \hline
    $TK_6$              & GEPLANT         & -50                   & 5                      & false                        & Plätze unter min.              & 1, 4, 6, 10 \\
    \hline
    $TK_7$              & GEPLANT         & 501                   & 3                      & false                        & Plätze edgecase max.           & 1, 5, 6, 9  \\
    \hline
    $ TK_8 $            & GEPLANT         & 600                   & 4                      & false                        & Plätze über max.               & 1, 5, 6, 9  \\
    \hline
    $TK_9$              & GEPLANT         & 400                   & 0                      & false                        & anzahlTickets edgecase min.    & 1, 3, 7, 9  \\
    \hline
    $ TK_{10}$          & GEPLANT         & 400                   & -25                    & false                        & anzahlTickets unter min.       & 1, 3, 7, 9  \\
    \hline
    $ TK_{11}$          & GEPLANT         & 400                   & 7                      & false                        & anzahlTickets edgecase max.    & 1, 3, 8, 9  \\
    \hline
    $TK_{12}$           & GEPLANT         & 4                     & 5                      & false                        & freiePlaetze < anzahlTickets   & 1, 3, 6, 10 \\
    \hline
  \end{tabular}
\end{landscape}

\begin{landscape}
  \exercise{Zustandsbasierter Test – Heißgetränkeautomat}
  \begin{enumerate}[label=\alph*)]
    \item ~ \\
          \begin{tikzpicture}[
              % Global style configuration
              node distance = 4cm and 5cm, % Vertical and Horizontal spacing
              box/.style = {
                  draw,
                  rectangle,
                  thick,
                  minimum width=3cm,
                  minimum height=1.5cm,
                  align=center,
                  font=\sffamily
                },
              arrow/.style = {
                  ->,
                  >=Latex,
                  thick,
                  rounded corners
                },
              label text/.style = {
                  midway,
                  font=\sffamily\small,
                  align=center
                }
            ]
            % --- Nodes (States) ---

            % 1. Ausgeschaltet
            \node[box] (off) {Ausgeschaltet};

            % 2. Betriebsbereit (to the right of Off)
            \node[box, right=of off] (ready) {Betriebsbereit};

            % 3. Kredit zustand (to the right of Ready)
            \node[box, right=of ready] (credit) {Kreditzustand};

            % 4. Zubereitung (CENTERED below Ready and Credit)
            % Logic: Take the midpoint (!0.5!) between ready and credit, then move down 3cm
            \node[box] (prep) at ($(ready.south)!0.5!(credit.south) + (0, -4cm)$) {Zubereitung};

            % --- Edges (Transitions) ---

            % Start Arrow
            \draw[arrow] ($(off.west) + (-1.5, 0)$) -- (off.west) node[] {};

            % Off <-> Ready (Parallel straight lines)
            \draw[arrow] ([yshift=2mm]off.east) -- ([yshift=2mm]ready.west)
            node[label text, above] {einschalten /\\ Anzeige Münzeinwurf,\\ Warten auf Münzeinwurf};
            \draw[arrow] ([yshift=-2mm]ready.west) -- ([yshift=-2mm]off.east)
            node[label text, below] {ausschalten};

            % Ready <-> Credit (Parallel straight lines)
            \draw[arrow] ([yshift=2mm]ready.east) -- ([yshift=2mm]credit.west)
            node[label text, above] {Münzeinwurf /\\ Warten auf Getränkewahl};
            \draw[arrow] ([yshift=-2mm]credit.west) -- ([yshift=-2mm]ready.east)
            node[label text, below] {Vorgang abbrechen /\\ Kreditrückgabe,\\ Warten auf Münzeinwurf};

            % Credit -> Prep (Diagonal down)
            \draw[arrow] (credit) -- (prep)
            node[label text, sloped, below, xshift=0.3cm] {Getränkeauswahl /\\ Tee- oder Kaffee- \\zubereitung starten};

            % Prep -> Ready (Diagonal up)
            \draw[arrow] (prep) -- (ready)
            node[label text, sloped, below, xshift=-0.3cm] {Getränk fertig,\\ Getränkeausgabe /\\ Warten auf Münzeinwurf};

            % Credit -> Off (Curved top)
            \draw[arrow] (credit.north) .. controls +(up:2.5cm) and +(up:2.5cm) .. (off.north)
            node[label text, above] {ausschalten};

            % Prep -> Off (Curved bottom)
            \draw[arrow] (prep.south) .. controls +(down:1.5cm) and +(down:3cm) .. (off.south)
            node[label text, sloped, below] {ausschalten};

          \end{tikzpicture}
          \newpage

    \item ~ \\
          \begin{tabular}{|p{3.35cm}|p{4cm}|p{4cm}|p{4cm}|p{4cm}|}
            \hline
            \diagbox{Ereignis}{Zustand}
             & Ausgeschaltet
             & Betriebsbereit
             & Kreditzustand
             & Zubereitung                                   \\
            \hline

            % Row 1: einschalten
            einschalten
             & { \raggedright \emph{Betriebsbereit} \par }
            { \raggedleft \footnotesize Anzeige Münzeinwurf, \\ warte auf Münzeinwurf \par }
             & \emph{Betriebsbereit}
             & \emph{Kreditzustand}
             & \emph{Zubereitung}                            \\
            \hline

            % Row 2: Münzeinwurf
            Münzeinwurf
             & \emph{Ausgeschaltet}
             & { \raggedright \emph{Kreditzustand} \par }
            { \raggedleft \footnotesize Warten auf           \\ Getränkeauswahl \par }
             & \emph{Kreditzustand}
             & \emph{Zubereitung}                            \\
            \hline

            % Row 3: Getränkeauswahl
            Getränkeauswahl
             & \emph{Ausgeschaltet}
             & \emph{Betriebsbereit}
             & { \raggedright \emph{Zubereitung} \par }
            { \raggedleft \footnotesize Tee- oder Kaffee-    \\ zubereitung starten \par }
             & Zubereitung                                   \\
            \hline

            % Row 4: Getränk fertig...
            Getränk fertig und Getränkeausgabe
             & \emph{Ausgeschaltet}
             & \emph{Betriebsbereit}
             & \emph{Kreditzustand}
             & { \raggedright \emph{Betriebsbereit} \par }
            { \raggedleft \footnotesize Warten auf           \\ Münzeinwurf \par } \\
            \hline

            % Row 5: Vorgang abbrechen
            Vorgang \newline abbrechen
             & \emph{Ausgeschaltet}
             & \emph{Betriebsbereit}
             & { \raggedright \emph{Betriebsbereit} \par }
            { \raggedleft \footnotesize Kreditrückgabe,      \\ Warten auf Münzeinwurf \par }
             & \emph{Zubereitung}                            \\
            \hline

            % Row 6: ausschalten
            ausschalten
             & \emph{Ausgeschaltet}
             & \emph{Ausgeschaltet}
             & \emph{Ausgeschaltet}
             & \emph{Ausgeschaltet}                          \\
            \hline
          \end{tabular}
  \end{enumerate}
\end{landscape}
\begin{enumerate}
  \item [c)] {
        Um eine vollständige Zustandsüberdeckung zu erreichen, müssen alle Zustände mindestens einmal besucht werden. In diesem Fall lassen sich innerhalb eines Durchlaufs alle Zustände besuchen, also reicht ein einziger Testfall aus, um alle Zustände abzudecken. Dieser sieht wie folgt aus:
        \begin{itemize}
          \item Ausgeschaltet \textarrow{einschalten} Betriebsbereit
          \item Betriebsbereit \textarrow{Münzeinwurf} Kreditzustand
          \item Kreditzustand \textarrow{Getränkeauswahl} Zubereitung
          \item Zubereitung \textarrow{ausschalten} Ausgeschaltet
        \end{itemize}
        So werden die Zustände Ausgeschaltet, Betriebsbereit, Kreditzustand und Zubereitung jeweils einmal besucht und eine vollständige Zustandsüberdeckung erreicht. Der Test ist erfolgreich, wenn alle Zustände erreicht werden können, andernfalls schlägt er fehl.
        }

  \item [d)] {
        Um eine vollständige Zustandsübergangsüberdeckung zu erreichen, müssen alle möglichen Übergänge zwischen den Zuständen mindestens einmal durchlaufen werden. In diesem Fall gibt es insgesamt 8 Übergänge, die abgedeckt werden müssen. Drei mögliche Testfälle, die zusammen alle Übergänge abdecken, sehen wie folgt aus:
        \begin{enumerate}[label=\roman*.]
          \item \begin{itemize}
                  \item Ausgeschaltet \textarrow{einschalten} Betriebsbereit
                  \item Betriebsbereit \textarrow{Münzeinwurf} Kreditzustand
                  \item Kreditzustand \textarrow{Getränkeauswahl} Zubereitung
                  \item Zubereitung \textarrow{Getränk fertig, Getränkeausgabe} Betriebsbereit
                  \item Betriebsbereit \textarrow{Münzeinwurf} Kreditzustand
                  \item Kreditzustand \textarrow{Vorgang abbrechen} Betriebsbereit
                  \item Betriebsbereit \textarrow{ausschalten} Ausgeschaltet \\
                \end{itemize}
          \item \begin{itemize}
                  \item Ausgeschaltet \textarrow{einschalten} Betriebsbereit
                  \item Betriebsbereit \textarrow{Münzeinwurf} Kreditzustand
                  \item Kreditzustand \textarrow{ausschalten} Ausgeschaltet \\
                \end{itemize}

          \item \begin{itemize}
                  \item Ausgeschaltet \textarrow{einschalten} Betriebsbereit
                  \item Betriebsbereit \textarrow{ausschalten} Ausgeschaltet
                  \item Ausgeschaltet \textarrow{einschalten} Betriebsbereit
                  \item Betriebsbereit \textarrow{Münzeinwurf} Kreditzustand
                  \item Kreditzustand \textarrow{Getränkeauswahl} Zubereitung
                  \item Zubereitung \textarrow{ausschalten} Ausgeschaltet \\
                \end{itemize}
        \end{enumerate}
        Das Testszenario ist erfolgreich, wenn alle Übergänge mindestens einmal durchlaufen wurden, andernfalls schlägt es fehl. Sprich, wenn ein Testfall fehlschlägt, schlägt das gesamte Szenario fehl.
        }
        \newpage
  \item [e)] {
        Um eine vollständige Ereignisüberdeckung zu erreichen, müssen alle möglichen Ereignisse mindestens einmal ausgelöst werden. Um dies zu automatisieren, nutzen wir eine modifizierte Tiefensuche. Dazu nehmen wir uns eine Datenstruktur zur Hilfe, welche alle bereits ausgelösten Ereignisse speichert.

        Die Tiefensuche wird so modifiziert, dass sie bei der Verfolgung eines Pfades nur Zweige in Betracht zieht, die ein bisher noch nicht ausgelöstes Ereignis enthalten. Sobald ein Pfad endet, weil alle möglichen Übergänge zu bereits ausgelösten Ereignissen führen, müsste die Suche zurückgehen und einen anderen Pfad verfolgen. Da dies jedoch nicht so einfach möglich ist, wird stattdessen ein neuer Testfall gestartet, der am Anfangszustand beginnt und erneut eine Tiefensuche durchführt, dieses Mal aber den zunächst zu betrachtenden Übergang auf seinem Pfad wählt.

        Dies wird so lange wiederholt, bis alle Ereignisse mindestens einmal ausgelöst wurden. Dieser Algorithmus funktioniert allerdings nur für zusammenhängende Zustandsautomaten, bei denen gewährleistet ist, dass sich alle Ereignisse von einem gemeinsamen Startzustand aus erreichen lassen. Sollte es mehrere Startzustände geben, würde der Algorithmus für jeden Startzustand separat ausgeführt werden müssen, was aber zu keinem Problem führt.

        Das Testszenario ist erfolgreich, wenn die Datenstruktur am Ende alle Ereignisse enthält, andernfalls schlägt es fehl.
        }
\end{enumerate}

\newpage
\exercise{Kontrollflussorientierter Test – Rabattberechnung}
\begin{enumerate}
  \setcounter{enumi}{1}
  \begin{figure}[H]
    \centering
    \includegraphics[width=1\linewidth]{./Tex-Graphics/EX03A31.png}
    \caption{Kontrollflussgraph}
    \label{fig:k1}
  \end{figure}
  \item[b)]
        \begin{itemize}
          \item Pfad 1(if Bedingung ist wahr.): $n_{\text{start}} \rightarrow n_1 \rightarrow n_{\text{final}}$
          \item Pfad 2(else if Bedingung ist wahr.): $n_{\text{start}} \rightarrow n_1 \rightarrow n_2 \rightarrow n_{\text{final}}$
          \item Pfad 3(keins von beidem ist wahr.): $n_{\text{start}} \rightarrow n_1 \rightarrow n_3 \rightarrow n_{\text{final}}$
        \end{itemize}
  \item[c)]
        Testfall 1: \\
        Eingabe: alter 17, mitglied false, student true.\\
        Ausgabe: true\\
        Testfall 2: \\
        Eingabe: alter 25, mitglied true, student false. \\
        Ausgabe: true \\

        Testfall 3: \\
        Eingabe: alter 20, mitglied false, student false \\
        Ausgabe: false \\

        Anweisungsüberdeckung und Zweigüberdeckung werden bei dieser Methode über die selbe minimale Menge an Testfällen erreicht.\\

  \item[d)]
        $n_{\text{start}} \rightarrow n_1 \rightarrow n_2 \rightarrow n_\text{final}$ ist ein theoretischer Pfad bei dem beide Bedinungen zutreffen. Dieser ist nicht ausführbar, da else if die zweite Bedingung nur dann Prüft, wenn die erste falsch ist.
\end{enumerate}
\end{document}
