% LaTeX Template für Abgaben an der Universität Stuttgart
% Autor: Sandro Speth
% Bei Fragen: Sandro.Speth@iste.uni-stuttgart.de
%-----------------------------------------------------------
% Hauptmodul des Templates: Hier können andere Dateien eingebunden werden
% oder Inhalte direkt rein geschrieben werden.
% Kompiliere dieses Modul um eine PDF zu erzeugen.

% Dokumentenart. Ersetze 12pt, falls die Schriftgröße anzupassen ist.
\documentclass[12pt]{scrartcl}
% Einbinden der Pakete, des Headers und der Formatierung.
% Mit den \include und \input Befehlen können Dateien eingebunden werden:
% \include: Fügt einen Seitenumbruch nach dem Text ein
% \input: Fügt KEINEN Seitenumbruch nach dem Text ein
\usepackage[utf8]{inputenc}
\usepackage[T1]{fontenc}
\usepackage[ngerman]{babel}
\usepackage{csquotes}
\usepackage{lmodern}
\usepackage{graphicx}
\usepackage{float}
\usepackage[pdftex,hyperref,dvipsnames]{xcolor}
\usepackage{listings}
\usepackage[a4paper,lmargin={2cm},rmargin={2cm},tmargin={3.5cm},bmargin = {2.5cm},headheight = {4cm}]{geometry}
\usepackage{amsmath,amssymb,amstext,amsthm}
\usepackage[lined,algonl,boxed]{algorithm2e}
\usepackage{tikz}
\usepackage{hyperref}
\usepackage{url}
\usepackage[inline]{enumitem} % Ermöglicht ändern der enum Item Zahlen
\usepackage[headsepline]{scrlayer-scrpage}
\usepackage{booktabs}
\usepackage{multirow}
\usepackage{pdflscape}
\usepackage{diagbox}
\pagestyle{scrheadings}
\usetikzlibrary{automata,positioning}
% LaTeX Template für Abgaben an der Universität Stuttgart
% Autor: Sandro Speth
% Bei Fragen: Sandro.Speth@iste.uni-stuttgart.de
%-----------------------------------------------------------
% Modul beinhaltet Befehl fuer Aufgabennummerierung,
% sowie die Header Informationen.

\providecommand{\BlattNr}{0}
\providecommand{\AuthorsNames}{keine Autoren gesetzt}

% Überschreibt enumerate Befehl, sodass 1. Ebene Items mit
\renewcommand{\theenumi}{(\alph{enumi})}
% (a), (b), etc. nummeriert werden.
\renewcommand{\labelenumi}{\text{\theenumi}}

% Counter für das Blatt und die Aufgabennummer.
% Ersetze die Nummer des Übungsblattes und die Nummer der Aufgabe
% den Anforderungen entsprechend.
% Gesetz werden die counter in der hauptdatei, damit siese hier nicht jedes mal verändert werden muss
% Beachte:
% \setcounter{countername}{number}: Legt den Wert des Counters fest
% \stepcounter{countername}: Erhöht den Wert des Counters um 1.
\newcounter{exnum}

% Befehl für die Aufgabentitel
\newcommand{\exercise}[1]{\section*{Aufgabe \theexnum\stepcounter{exnum}: #1}} % Befehl für Aufgabentitel

% Formatierung der Kopfzeile
% \ohead: Setzt rechten Teil der Kopfzeile mit
% Namen und Matrikelnummern aller Bearbeiter
\ohead {\AuthorsNames}
% \chead{} kann mittleren Kopfzeilen Teil sezten
% \ihead: Setzt linken Teil der Kopfzeile mit
% Modulnamen, Semester und Übungsblattnummer
\ihead{Advanced Software Engineering\\
Wintersemester 2025\\
Übungsblatt \BlattNr}

\setcounter{exnum}{1} % Nummer der Aufgabe

\begin{document}

    \providecommand{\BlattNr}{0}
\providecommand{\AuthorsNames}{keine Authoren gesetzt}

\begin{titlepage}
	\centering
	\includegraphics[width=0.35\textwidth]{../assets/icon_ase.png}\par\vspace{1cm}
	{\scshape\LARGE Universität Stuttgart \par}
	\vspace{1cm}
	{\scshape\Large Advanced Software Engineering\par}
	\vspace{1.5cm}
	{\huge\bfseries Übungsblatt \BlattNr \par}
	\vspace{2cm}
	{\Large \AuthorsNames \par}
	\vfill
    Wintersemester 2025
    \vfill
	{\large \today\par}
	\end{titlepage}
 % Titelseite einbinden

% Nutze den \exercise{Aufgabenname} Befehl, um eine neue Aufgabe zu beginnen.
% Möchtest du eine Aufgabe in der Nummerierung überspringen, schreibe vor der Aufgabe: \stepcounter{exnum}
% Möchtest du die Nummer einer Aufgabe auf eine beliebige Zahl x setzen, schreibe vor der Aufgabe: \setcounter{exnum}{x}

% Aufgabe 1
    \exercise{Kritischer Pfad}

    \begin{enumerate}
        \item[(a)]{

\begin{figure}[h]
    \centering
    \includegraphics[width=1.0\textwidth]{./Tex-Graphics/Aufgabe1a.jpeg}
    \caption{Instanz der CPM}
\end{figure}
        }
        \item[(c)]{Der verwendete Algorithmus besteht im westentlichen zwei Phasen. In der ersten Phase, der Forward-Propagation, werden
        zunächst die Startknoten identifiziert. Dies sind alle Knoten ohne Abhängigkeiten. Für diese wird dann EF als ES+duration berechnet. Diese neue Informationen werden an alle Knoten mit direkten Abhängigkeiten weitergegeben. Dies wiederholt sich rekursiv, bis alle Knoten den Zeitpunkt EF erhalten haben. In der darauf folgenden zweiten Phase wird die Gesamtdauer des Projekt als Maximum aller EF berechnet und als LF aller Knoten ohne Nachfolger (Blattknoten) gesetzt. Daraus ergibt sich für alle Blattknoten der LS als LF-duration. Diese Informationen werden dann an die Eltern (alle Knoten von denen der aktuelle Knoten abhängt) rekursiv weitergegeben, bis schlussendlich alle Knoten ES,EF,LS,LF ermittelt haben. Zum Schluss wird für jeden Knoten der Slack als LS-ES = LF-EF berechnet. 
        \\
        Um eine bessere Softwarequalität zu erreichen, wurde entscheiden eine Versionskontrolle (git) zu verwenden. Die ist hilfreich, wenn mehrere Entwickler am selben Projekt arbeiten möchten, da alle Änderungen getrackt werden und notfalls auch rückgänging gemacht werden können. Desweiteren wird GitHubActions als CI verwendet, um u.a. den Main-Branch sauber zu halten. Dadurch sollte gewährleistet sein, dass immer eine ausführbare Version der Software verfügbar ist.
        }
    \end{enumerate}


\end{document}